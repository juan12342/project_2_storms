\documentclass[]{article}
\usepackage{lmodern}
\usepackage{amssymb,amsmath}
\usepackage{ifxetex,ifluatex}
\usepackage{fixltx2e} % provides \textsubscript
\ifnum 0\ifxetex 1\fi\ifluatex 1\fi=0 % if pdftex
  \usepackage[T1]{fontenc}
  \usepackage[utf8]{inputenc}
\else % if luatex or xelatex
  \ifxetex
    \usepackage{mathspec}
  \else
    \usepackage{fontspec}
  \fi
  \defaultfontfeatures{Ligatures=TeX,Scale=MatchLowercase}
\fi
% use upquote if available, for straight quotes in verbatim environments
\IfFileExists{upquote.sty}{\usepackage{upquote}}{}
% use microtype if available
\IfFileExists{microtype.sty}{%
\usepackage{microtype}
\UseMicrotypeSet[protrusion]{basicmath} % disable protrusion for tt fonts
}{}
\usepackage[margin=1in]{geometry}
\usepackage{hyperref}
\hypersetup{unicode=true,
            pdftitle={STORMPROJECT},
            pdfauthor={Juan David Rico Velasco},
            pdfborder={0 0 0},
            breaklinks=true}
\urlstyle{same}  % don't use monospace font for urls
\usepackage{color}
\usepackage{fancyvrb}
\newcommand{\VerbBar}{|}
\newcommand{\VERB}{\Verb[commandchars=\\\{\}]}
\DefineVerbatimEnvironment{Highlighting}{Verbatim}{commandchars=\\\{\}}
% Add ',fontsize=\small' for more characters per line
\usepackage{framed}
\definecolor{shadecolor}{RGB}{248,248,248}
\newenvironment{Shaded}{\begin{snugshade}}{\end{snugshade}}
\newcommand{\KeywordTok}[1]{\textcolor[rgb]{0.13,0.29,0.53}{\textbf{#1}}}
\newcommand{\DataTypeTok}[1]{\textcolor[rgb]{0.13,0.29,0.53}{#1}}
\newcommand{\DecValTok}[1]{\textcolor[rgb]{0.00,0.00,0.81}{#1}}
\newcommand{\BaseNTok}[1]{\textcolor[rgb]{0.00,0.00,0.81}{#1}}
\newcommand{\FloatTok}[1]{\textcolor[rgb]{0.00,0.00,0.81}{#1}}
\newcommand{\ConstantTok}[1]{\textcolor[rgb]{0.00,0.00,0.00}{#1}}
\newcommand{\CharTok}[1]{\textcolor[rgb]{0.31,0.60,0.02}{#1}}
\newcommand{\SpecialCharTok}[1]{\textcolor[rgb]{0.00,0.00,0.00}{#1}}
\newcommand{\StringTok}[1]{\textcolor[rgb]{0.31,0.60,0.02}{#1}}
\newcommand{\VerbatimStringTok}[1]{\textcolor[rgb]{0.31,0.60,0.02}{#1}}
\newcommand{\SpecialStringTok}[1]{\textcolor[rgb]{0.31,0.60,0.02}{#1}}
\newcommand{\ImportTok}[1]{#1}
\newcommand{\CommentTok}[1]{\textcolor[rgb]{0.56,0.35,0.01}{\textit{#1}}}
\newcommand{\DocumentationTok}[1]{\textcolor[rgb]{0.56,0.35,0.01}{\textbf{\textit{#1}}}}
\newcommand{\AnnotationTok}[1]{\textcolor[rgb]{0.56,0.35,0.01}{\textbf{\textit{#1}}}}
\newcommand{\CommentVarTok}[1]{\textcolor[rgb]{0.56,0.35,0.01}{\textbf{\textit{#1}}}}
\newcommand{\OtherTok}[1]{\textcolor[rgb]{0.56,0.35,0.01}{#1}}
\newcommand{\FunctionTok}[1]{\textcolor[rgb]{0.00,0.00,0.00}{#1}}
\newcommand{\VariableTok}[1]{\textcolor[rgb]{0.00,0.00,0.00}{#1}}
\newcommand{\ControlFlowTok}[1]{\textcolor[rgb]{0.13,0.29,0.53}{\textbf{#1}}}
\newcommand{\OperatorTok}[1]{\textcolor[rgb]{0.81,0.36,0.00}{\textbf{#1}}}
\newcommand{\BuiltInTok}[1]{#1}
\newcommand{\ExtensionTok}[1]{#1}
\newcommand{\PreprocessorTok}[1]{\textcolor[rgb]{0.56,0.35,0.01}{\textit{#1}}}
\newcommand{\AttributeTok}[1]{\textcolor[rgb]{0.77,0.63,0.00}{#1}}
\newcommand{\RegionMarkerTok}[1]{#1}
\newcommand{\InformationTok}[1]{\textcolor[rgb]{0.56,0.35,0.01}{\textbf{\textit{#1}}}}
\newcommand{\WarningTok}[1]{\textcolor[rgb]{0.56,0.35,0.01}{\textbf{\textit{#1}}}}
\newcommand{\AlertTok}[1]{\textcolor[rgb]{0.94,0.16,0.16}{#1}}
\newcommand{\ErrorTok}[1]{\textcolor[rgb]{0.64,0.00,0.00}{\textbf{#1}}}
\newcommand{\NormalTok}[1]{#1}
\usepackage{graphicx,grffile}
\makeatletter
\def\maxwidth{\ifdim\Gin@nat@width>\linewidth\linewidth\else\Gin@nat@width\fi}
\def\maxheight{\ifdim\Gin@nat@height>\textheight\textheight\else\Gin@nat@height\fi}
\makeatother
% Scale images if necessary, so that they will not overflow the page
% margins by default, and it is still possible to overwrite the defaults
% using explicit options in \includegraphics[width, height, ...]{}
\setkeys{Gin}{width=\maxwidth,height=\maxheight,keepaspectratio}
\IfFileExists{parskip.sty}{%
\usepackage{parskip}
}{% else
\setlength{\parindent}{0pt}
\setlength{\parskip}{6pt plus 2pt minus 1pt}
}
\setlength{\emergencystretch}{3em}  % prevent overfull lines
\providecommand{\tightlist}{%
  \setlength{\itemsep}{0pt}\setlength{\parskip}{0pt}}
\setcounter{secnumdepth}{0}
% Redefines (sub)paragraphs to behave more like sections
\ifx\paragraph\undefined\else
\let\oldparagraph\paragraph
\renewcommand{\paragraph}[1]{\oldparagraph{#1}\mbox{}}
\fi
\ifx\subparagraph\undefined\else
\let\oldsubparagraph\subparagraph
\renewcommand{\subparagraph}[1]{\oldsubparagraph{#1}\mbox{}}
\fi

%%% Use protect on footnotes to avoid problems with footnotes in titles
\let\rmarkdownfootnote\footnote%
\def\footnote{\protect\rmarkdownfootnote}

%%% Change title format to be more compact
\usepackage{titling}

% Create subtitle command for use in maketitle
\providecommand{\subtitle}[1]{
  \posttitle{
    \begin{center}\large#1\end{center}
    }
}

\setlength{\droptitle}{-2em}

  \title{STORMPROJECT}
    \pretitle{\vspace{\droptitle}\centering\huge}
  \posttitle{\par}
    \author{Juan David Rico Velasco}
    \preauthor{\centering\large\emph}
  \postauthor{\par}
      \predate{\centering\large\emph}
  \postdate{\par}
    \date{30 de agosto de 2020}


\begin{document}
\maketitle

\subsection{R Markdown}\label{r-markdown}

\section{Synopsis}\label{synopsis}

Storms and other severe weather events have huge impact on public health
and economic problems for municipalities and their inhabitants. Some of
severe events can cause injuries property damage and even lead to death.
This analysis present which types of events are most harmful with
respect to population health and which have the greatest economic
consequences.

\section{Data Processing}\label{data-processing}

I'm going to use The U.S. National Oceanic and Atmospheric
Administration's (NOAA) storm database which tracks characteristics of
major storms and weather events in the United States. This dataset comes
from the Internet.

Download file from the Internet:

\begin{Shaded}
\begin{Highlighting}[]
\NormalTok{link <-}\StringTok{ "http://d396qusza40orc.cloudfront.net/repdata%2Fdata%2FStormData.csv.bz2"}
\KeywordTok{download.file}\NormalTok{(}\DataTypeTok{url =}\NormalTok{ link, }\DataTypeTok{destfile =} \StringTok{"StormData"}\NormalTok{)}
\end{Highlighting}
\end{Shaded}

Read a file in table format

Property damage estimates were entered as actual dollar amounts (the
variable PROPDMG). But they were rounded to three significant digits,
followed by an alphabetical character signifying the magnitude of the
number, i.e., 1.55B for \$1,550,000,000. Alphabetical characters used to
signify magnitude include ``K'' for thousands, ``M'' for millions, and
``B'' for billions. So I created a new variable PROPDMGEXP2 and assigned
conditionally ``K'' = 1000, ``M'' = 1000000, ``B'' = 1000000000, in
other cases 1. These variables are multiplied in the next step.

\begin{Shaded}
\begin{Highlighting}[]
\KeywordTok{table}\NormalTok{(StormData}\OperatorTok{$}\NormalTok{PROPDMGEXP)}
\end{Highlighting}
\end{Shaded}

\begin{verbatim}
## 
##      -      ?      +      0      1      2      3      4      5      6 
##      1      8      5    216     25     13      4      4     28      4 
##      7      8      B      h      H      K      m      M 
##      5      1     40      1      6 424665      7  11330
\end{verbatim}

\begin{Shaded}
\begin{Highlighting}[]
\NormalTok{StormData}\OperatorTok{$}\NormalTok{PROPDMGEXP2 <-}\StringTok{ }\DecValTok{1}
\NormalTok{StormData}\OperatorTok{$}\NormalTok{PROPDMGEXP2[}\KeywordTok{which}\NormalTok{(StormData}\OperatorTok{$}\NormalTok{PROPDMGEXP }\OperatorTok{==}\StringTok{ "K"}\NormalTok{)] <-}\StringTok{ }\DecValTok{1000}
\NormalTok{StormData}\OperatorTok{$}\NormalTok{PROPDMGEXP2[}\KeywordTok{which}\NormalTok{(StormData}\OperatorTok{$}\NormalTok{PROPDMGEXP }\OperatorTok{==}\StringTok{ "M"} \OperatorTok{|}\StringTok{ }\NormalTok{StormData}\OperatorTok{$}\NormalTok{PROPDMGEXP }\OperatorTok{==}\StringTok{ "m"}\NormalTok{)] <-}\StringTok{ }\DecValTok{1000000}
\NormalTok{StormData}\OperatorTok{$}\NormalTok{PROPDMGEXP2[}\KeywordTok{which}\NormalTok{(StormData}\OperatorTok{$}\NormalTok{PROPDMGEXP }\OperatorTok{==}\StringTok{ "B"}\NormalTok{)] <-}\StringTok{ }\DecValTok{1000000000}
\end{Highlighting}
\end{Shaded}

\begin{Shaded}
\begin{Highlighting}[]
\KeywordTok{table}\NormalTok{(StormData}\OperatorTok{$}\NormalTok{PROPDMGEXP2)}
\end{Highlighting}
\end{Shaded}

\begin{verbatim}
## 
##      1   1000  1e+06  1e+09 
## 466255 424665  11337     40
\end{verbatim}

\section{Which types of events are most harmful to population
health?}\label{which-types-of-events-are-most-harmful-to-population-health}

Fatalities and injuries have the most impact on public health, so I will
present what types of severe weather are the most dangerous.

The first plot presents a Death toll by Event type

\begin{Shaded}
\begin{Highlighting}[]
\NormalTok{StormData }\OperatorTok
\StringTok{      }\KeywordTok{select}\NormalTok{(FATALITIES, EVTYPE) }\OperatorTok
\StringTok{      }\KeywordTok{group_by}\NormalTok{(EVTYPE) }\OperatorTok
\StringTok{      }\KeywordTok{summarise}\NormalTok{(}\DataTypeTok{SumFATALITIES =} \KeywordTok{sum}\NormalTok{(FATALITIES)) }\OperatorTok
\StringTok{      }\KeywordTok{top_n}\NormalTok{(}\DataTypeTok{n =} \DecValTok{8}\NormalTok{, }\DataTypeTok{wt =}\NormalTok{ SumFATALITIES) }\OperatorTok
\StringTok{      }\KeywordTok{ggplot}\NormalTok{(}\KeywordTok{aes}\NormalTok{(}\DataTypeTok{y =}\NormalTok{ SumFATALITIES, }\DataTypeTok{x =} \KeywordTok{reorder}\NormalTok{(}\DataTypeTok{x =}\NormalTok{ EVTYPE, }\DataTypeTok{X =}\NormalTok{ SumFATALITIES), }\DataTypeTok{fill=}\NormalTok{EVTYPE))}\OperatorTok{+}
\StringTok{      }\KeywordTok{geom_bar}\NormalTok{(}\DataTypeTok{stat =} \StringTok{"identity"}\NormalTok{, }\DataTypeTok{show.legend =} \OtherTok{FALSE}\NormalTok{) }\OperatorTok{+}
\StringTok{      }\CommentTok{#geom_text(aes(label=SumFATALITIES), size = 4, hjust = 0.5, vjust = -0.1) +}
\StringTok{      }\KeywordTok{xlab}\NormalTok{(}\DataTypeTok{label =} \StringTok{""}\NormalTok{) }\OperatorTok{+}
\StringTok{      }\KeywordTok{ylab}\NormalTok{(}\DataTypeTok{label =} \StringTok{"Death toll"}\NormalTok{) }\OperatorTok{+}
\StringTok{      }\KeywordTok{coord_flip}\NormalTok{() }\OperatorTok{+}
\StringTok{      }\KeywordTok{theme_light}\NormalTok{()}
\end{Highlighting}
\end{Shaded}

\includegraphics{stormfinal_files/figure-latex/unnamed-chunk-5-1.pdf}

The second plot presents Injuries by Event type

\begin{Shaded}
\begin{Highlighting}[]
\NormalTok{StormData }\OperatorTok
\StringTok{      }\KeywordTok{select}\NormalTok{(INJURIES, EVTYPE) }\OperatorTok
\StringTok{      }\KeywordTok{group_by}\NormalTok{(EVTYPE) }\OperatorTok
\StringTok{      }\KeywordTok{summarise}\NormalTok{(}\DataTypeTok{SumINJURIES =} \KeywordTok{sum}\NormalTok{(INJURIES)) }\OperatorTok
\StringTok{      }\KeywordTok{top_n}\NormalTok{(}\DataTypeTok{n =} \DecValTok{8}\NormalTok{, }\DataTypeTok{wt =}\NormalTok{ SumINJURIES) }\OperatorTok
\StringTok{      }\KeywordTok{ggplot}\NormalTok{(}\KeywordTok{aes}\NormalTok{(}\DataTypeTok{y =}\NormalTok{ SumINJURIES, }\DataTypeTok{x =} \KeywordTok{reorder}\NormalTok{(}\DataTypeTok{x =}\NormalTok{ EVTYPE, }\DataTypeTok{X =}\NormalTok{ SumINJURIES), }\DataTypeTok{fill=}\NormalTok{EVTYPE))}\OperatorTok{+}
\StringTok{      }\KeywordTok{geom_bar}\NormalTok{(}\DataTypeTok{stat =} \StringTok{"identity"}\NormalTok{, }\DataTypeTok{show.legend =} \OtherTok{FALSE}\NormalTok{) }\OperatorTok{+}
\StringTok{      }\CommentTok{#geom_text(aes(label=SumINJURIES), size = 4, hjust = 0.5, vjust = -0.1) +}
\StringTok{      }\KeywordTok{xlab}\NormalTok{(}\DataTypeTok{label =} \StringTok{""}\NormalTok{) }\OperatorTok{+}
\StringTok{      }\KeywordTok{ylab}\NormalTok{(}\DataTypeTok{label =} \StringTok{"INJURIES"}\NormalTok{) }\OperatorTok{+}
\StringTok{      }\KeywordTok{coord_flip}\NormalTok{() }\OperatorTok{+}
\StringTok{      }\KeywordTok{theme_light}\NormalTok{()}
\end{Highlighting}
\end{Shaded}

\includegraphics{stormfinal_files/figure-latex/unnamed-chunk-6-1.pdf}

\section{Which types of events have the greatest economic
consequences?}\label{which-types-of-events-have-the-greatest-economic-consequences}

This plot shows Property damage estimates by Event type

\begin{Shaded}
\begin{Highlighting}[]
\NormalTok{StormData }\OperatorTok
\StringTok{      }\KeywordTok{select}\NormalTok{(PROPDMG, PROPDMGEXP2, EVTYPE) }\OperatorTok
\StringTok{      }\KeywordTok{group_by}\NormalTok{(EVTYPE) }\OperatorTok
\StringTok{      }\KeywordTok{mutate}\NormalTok{(}\DataTypeTok{SumPROPDMGEXP =}\NormalTok{ (PROPDMG }\OperatorTok{*}\StringTok{ }\NormalTok{PROPDMGEXP2)) }\OperatorTok
\StringTok{      }\KeywordTok{summarise}\NormalTok{(}\DataTypeTok{SumPROPDMGEXP2 =} \KeywordTok{sum}\NormalTok{(SumPROPDMGEXP)) }\OperatorTok
\StringTok{      }\KeywordTok{top_n}\NormalTok{(}\DataTypeTok{n =} \DecValTok{8}\NormalTok{, }\DataTypeTok{wt =}\NormalTok{ SumPROPDMGEXP2) }\OperatorTok
\StringTok{      }\KeywordTok{ggplot}\NormalTok{(}\KeywordTok{aes}\NormalTok{(}\DataTypeTok{y =}\NormalTok{ SumPROPDMGEXP2, }\DataTypeTok{x =} \KeywordTok{reorder}\NormalTok{(}\DataTypeTok{x =}\NormalTok{ EVTYPE, }\DataTypeTok{X =}\NormalTok{ SumPROPDMGEXP2), }\DataTypeTok{fill=}\NormalTok{EVTYPE))}\OperatorTok{+}
\StringTok{      }\KeywordTok{geom_bar}\NormalTok{(}\DataTypeTok{stat =} \StringTok{"identity"}\NormalTok{, }\DataTypeTok{show.legend =} \OtherTok{FALSE}\NormalTok{) }\OperatorTok{+}
\StringTok{      }\CommentTok{#geom_text(aes(label=SumFATALITIES), size = 4, hjust = 0.5, vjust = -0.1) +}
\StringTok{      }\KeywordTok{xlab}\NormalTok{(}\DataTypeTok{label =} \StringTok{""}\NormalTok{) }\OperatorTok{+}
\StringTok{      }\KeywordTok{ylab}\NormalTok{(}\DataTypeTok{label =} \StringTok{"Property damage estimates"}\NormalTok{) }\OperatorTok{+}
\StringTok{      }\KeywordTok{coord_flip}\NormalTok{() }\OperatorTok{+}
\StringTok{      }\KeywordTok{theme_light}\NormalTok{()}
\end{Highlighting}
\end{Shaded}

\includegraphics{stormfinal_files/figure-latex/unnamed-chunk-7-1.pdf}

\section{Conclusion}\label{conclusion}

\paragraph{As you can see above flood has the greatest economic
consequences. Tornado is the most harmful to population health because
caused the most death tolls and
injuries.}\label{as-you-can-see-above-flood-has-the-greatest-economic-consequences.-tornado-is-the-most-harmful-to-population-health-because-caused-the-most-death-tolls-and-injuries.}


\end{document}
